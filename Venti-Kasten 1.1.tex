\documentclass[11pt,a4paper]{article}
\usepackage[utf8]{inputenc}
\usepackage[T1]{fontenc}
\usepackage{lmodern}
\usepackage{geometry}
\usepackage{titlesec}
\usepackage{enumitem}
\usepackage{hyperref}

\geometry{margin=2.5cm}
\titleformat{\section}{\normalfont\Large\bfseries}{\thesection}{1em}{}
\titleformat{\subsection}{\normalfont\bfseries}{\thesubsection}{1em}{}

\title{\parbox{\linewidth}{\centering\textbf{
Projektidee: Der virtuelle Naturbaum\\
Adaptive Konvektionsl\"uftung f\"ur Bienenbeuten (Venti-Kasten 1.1)}}}

\author{Jens Buttensch\"on}
\date{April 2025 -- Version 1.1}

\begin{document}
\maketitle

\section*{Was ist neu in Version 1.1?}
\begin{itemize}[topsep=2pt]
  \item Erweiterung um Luftvolumenabsch\"atzung f\"ur unterschiedliche Beutenkonfigurationen.
  \item Integration einer Modellrechnung zum Wasserdampfaustrag pro Luftwechsel.
  \item Schwarmvermeidung durch schnellere Honigreifung und Platzoptimierung.
  \item Exakte Begriffskorrektur: \textbf{Segeberger Beute} statt "S\"ageberger".
  \item Strukturierung f\"ur besseres Pilotprojektmanagement.
  \item Diskussion offener L\"uftungskonzepte (Servo-Klappe vs. solarbetriebener Mikro-L\"ufter).
\end{itemize}

\section*{Leitgedanke: Der virtuelle Naturbaum}
In der Natur bevorzugen Bienen hohle B\"aume als Behausung, die durch Holzstruktur ein
stabiles Mikroklima bieten. Dieses Prinzip simulieren wir technologisch:
energieeffizient, ressourcenschonend, bienenfreundlich.

\section{Zielsetzung}
Entwicklung eines autarken Systems, das durch intelligente Konvektionsl\"uftung:
\begin{itemize}[topsep=2pt]
  \item den F\"achelaufwand der Bienen minimiert,
  \item die Honigreife beschleunigt,
  \item Temperaturschwankungen abfedert,
  \item ein naturnahes Baumklima nachbildet,
  \item und die Schwarmneigung aktiv reduziert.
\end{itemize}

\section{Technisches Prinzip: Der umfunktionierte Futterkasten}
\begin{itemize}[topsep=2pt]
  \item \textbf{Sensorik:} Temperatur- und Luftfeuchtesensoren innen und au\ss{}en.
  \item \textbf{Aktoreinheit:} Servo-Klappe im modifizierten Futterkasten.
  \item \textbf{Logik:} Mikrocontroller (z.B. ESP32) mit adaptiver Regelsoftware.
  \item \textbf{Energie:} Versorgung durch PV-Panel und Akku.
\end{itemize}
\textbf{L\"uftungskonzept:} Aktuell werden verschiedene Varianten des Feuchteabtransports diskutiert. Neben einer servo-gesteuerten Abluftklappe steht auch der Einsatz eines solarbetriebenen Mikro-L\"ufters zur Auswahl, der tags\"uber automatisch Luftstrom erzeugt und nachts ruht. Die finale L\"osung wird basierend auf Energieverbrauch, Zuverl\"assigkeit und thermodynamischer Effizienz im Praxistest bestimmt.

\section{Adaptiver Algorithmus}
\begin{itemize}[topsep=2pt]
  \item Erfassung von Innen- und Au\ss{}enklimadaten.
  \item Bewertung der aktuellen Feuchte- und Temperaturdifferenzen.
  \item Prognose des potenziellen Wasserdampfaustrags.
  \item Entscheidungslogik: gezielte, schonende Luftwechsel.
\end{itemize}

\section{Erweiterungen in Version 1.1}
\subsection{1. Luftvolumenabsch\"atzung der Beute}
Basierend auf Segeberger Standardma\ss{}en:
\begin{itemize}[topsep=2pt]
  \item 1 Zarge: ca. 30--35 Liter Innenvolumen.
  \item 2 Zargen: ca. 60--70 Liter.
  \item 3 Zargen: ca. 90--100 Liter.
\end{itemize}

\subsection{2. Wasserdampfaustragsmodellierung}
Ermittlung der Feuchteabgabe pro Luftaustausch:
\begin{itemize}[topsep=2pt]
  \item Volumen \( \times \) (Unterschied absolute Feuchte innen/au\ss{}en).
  \item Ziel: schnelle Entlastung der Waben vom eingetragenen Nektarwasser.
\end{itemize}

\subsection{3. Schwarmvermeidung durch Platzoptimierung}
\begin{itemize}[topsep=2pt]
  \item Schnellere Honigreifung schafft mehr Lagerraum.
  \item Entlastung verhindert Platznot im Brutraum.
  \item Reduzierter Schwarmdruck durch bessere Ressourcensteuerung.
\end{itemize}

\section{Integration mit Infrastruktur}
\begin{itemize}[topsep=2pt]
  \item Anbindung an PV-/Akku- und Mobilfunksysteme denkbar.
  \item Kombination mit Gewichtsdaten, Wetterprofilen oder Flugaktivit\"at optional m\"oglich.
  \item Sp\"ater denkbar: selbstoptimierende L\"uftungsmuster auf Basis externer Plattformen.
\end{itemize}

\section{Nachhaltigkeit und Wirkung}
\begin{itemize}[topsep=2pt]
  \item Nutzung nat\"urlicher Konvektionsmechanismen.
  \item Energieautarke, bienenfreundliche Optimierung.
  \item Anwendbar auf Segeberger Beuten und andere Systeme.
\end{itemize}

\section{N\"achste Schritte}
\begin{enumerate}[topsep=2pt]
  \item Entwicklung eines modularen Prototyps.
  \item Pilotierung an realen Bienenv\"olkern.
  \item Langzeitdatenerfassung und Optimierungsanalyse.
\end{enumerate}

\section*{Kontakt}
\textbf{Jens Buttensch\"on} \\
\small{\url{https://www.xing.com/profile/Jens_Buttenschoen}}

\end{document}
