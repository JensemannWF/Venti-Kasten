\documentclass[11pt,a4paper]{article}
\usepackage[utf8]{inputenc}
\usepackage[T1]{fontenc}
\usepackage{lmodern}
\usepackage{geometry}
\usepackage{titlesec}
\usepackage{enumitem}
\usepackage{hyperref}

\geometry{margin=2.5cm}
\titleformat{\section}{\normalfont\Large\bfseries}{\thesection}{1em}{}
\titleformat{\subsection}{\normalfont\bfseries}{\thesubsection}{1em}{}

\title{\parbox{\linewidth}{\centering\textbf{
Forschungsprotokoll Venti-Kasten 1.2\\
Pilotversuch zur adaptiven Konvektionsl\"uftung in Segeberger Bienenbeuten}}}

\author{Jens Buttensch\"on}
\date{April 2025}

\begin{document}
\maketitle

\section*{Zielsetzung}
Untersuchung der Effekte eines adaptiven Konvektionsl\"uftungssystems (Venti-Kasten) auf:
\begin{itemize}[topsep=2pt]
  \item Geschwindigkeit der Honigreife (Verdeckelung der Waben),
  \item Schwarmneigung der Bienenv\"olker,
  \item Energieverbrauch der Bienen (indirekt \"uber Reduktion von F\"achelaufwand).
\end{itemize}

\section*{Aufbau des Versuchs}
\subsection*{Versuchsgruppen}
\begin{itemize}[topsep=2pt]
  \item \textbf{Kontrollgruppe:} 1 Bienenvolk auf Standard-Segeberger Beute ohne aktive Klimasteuerung.
  \item \textbf{Testgruppe:} 1 Bienenvolk auf identischer Segeberger Beute mit aktivem Venti-Kasten.
\end{itemize}

\subsection*{Vorbedingungen}
\begin{itemize}[topsep=2pt]
  \item M\"oglichst Schwesterk\"oniginnen zur Minimierung genetischer Unterschiede.
  \item Vergleichbare Volksst\"arke und Ausgangsgewicht.
  \item Beide V\"olker stehen am gleichen Standort (gleiche Wetterbedingungen).
\end{itemize}

\section*{Mess- und Beobachtungsparameter}
\begin{itemize}[topsep=2pt]
  \item \textbf{Stockgewicht:} gemessen \"uber Stockwaagen, t\"aglich oder st\"undlich.
  \item \textbf{Innenklima:} Temperatur und Luftfeuchtigkeit innen, kontinuierlich aufgezeichnet.
  \item \textbf{Au\ss{}enklima:} Temperatur und Luftfeuchtigkeit au\ss{}en.
  \item \textbf{Honigreife:} Visuelle Pr\"ufung des Verdeckelungsgrads im Honigraum (alle 3--5 Tage).
  \item \textbf{Schwarmstimmung:} Beobachtung von Weiselzellen und Schwarmtrieb.
\end{itemize}

\section*{L\"uftungsstrategie im Testvolk}
\begin{itemize}[topsep=2pt]
  \item Aktive L\"uftung bei:
  \begin{itemize}[topsep=1pt]
    \item Innenluft rF > 70\%,
    \item Au\ss{}enluft trockener als Innenluft,
    \item Au\ss{}entemperatur > 18\,\textcelsius{} und < 30\,\textcelsius{}.
  \end{itemize}
  \item Abbruch der L\"uftung bei:
  \begin{itemize}[topsep=1pt]
    \item Innen-rF < 55\%,
    \item oder Innen-Temperatur < 33\,\textcelsius{}.
  \end{itemize}
\end{itemize}

\section*{Erwartete Beobachtungen}
\begin{itemize}[topsep=2pt]
  \item Schnellere Reduktion des Wassergehalts im Nektar im Testvolk.
  \item Fr\"uhere Verdeckelung der Honigr\"ahmchen.
  \item Stabilere Volksentwicklung ohne erh\"ohten Schwarmtrieb.
  \item M\"oglich geringere Gewichtszunahme aufgrund fr\"uher Honigreife.
\end{itemize}

\section*{Langfristige Perspektive}
Best\"atigung der Hypothese, dass adaptive Konvektionsl\"uftung zu:
\begin{itemize}[topsep=2pt]
  \item H\"oherer Bienengesundheit,
  \item Stabilerer Honigproduktion,
  \item und verl\"angerter V\"olkerstabilit\"at beitragen kann.
\end{itemize}

\section*{Kontakt}
\textbf{Projektleitung:} Jens Buttensch\"on \\
\small{\url{https://www.xing.com/profile/Jens_Buttenschoen}}

\end{document}
