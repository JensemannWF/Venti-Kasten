\documentclass[11pt,a4paper]{article}
\usepackage[utf8]{inputenc}
\usepackage[T1]{fontenc}
\usepackage[ngerman]{babel}
\usepackage{geometry}
\geometry{margin=2.5cm}
\usepackage{parskip}
\usepackage{hyperref}

\title{Kommunikationsidee Venti-Kasten – Intermodularer Datentransfer per Funk}
\author{Jens Buttenschön}
\date{Mai 2025}

\begin{document}

\maketitle

\section*{Ausgangsidee}
Die bisherige Planung des Venti-Kastens sieht eine zentrale Steuer- und Sensoreinheit im Deckel der Beute vor. Um jedoch die Verdrahtung zwischen Boden und Deckel zu vermeiden (z.\,B. zur Messung von Luftfeuchtigkeit, Temperatur, CO\textsubscript{2}), wurde die Idee entwickelt, den Deckel als \textbf{eigenständiges Modul mit eigenem ESP32} zu betreiben und die Daten per \textbf{Funk} an den Venti-Kasten im Bodenbereich zu übermitteln.

\section*{Technisches Konzept}
\begin{itemize}
  \item \textbf{Deckelmodul (Sender)}: 
  \begin{itemize}
    \item ESP32 mit Batteriespeisung (z.\,B. 3x AA, CR2032 oder Mini-Solarzelle)
    \item Sensorik für Luftfeuchte, Temperatur, ggf. CO\textsubscript{2}
    \item Übermittlung der Messdaten per \textbf{ESP-NOW} an den Venti-Kasten unten
  \end{itemize}
  \item \textbf{Bodenmodul (Empfänger)}:
  \begin{itemize}
    \item ESP32 mit Powerbank und ggf. Solar-Ladung
    \item Zentrale Steuerung der Lüfter und Aktorik
    \item Empfang der Daten vom Deckelmodul
    \item Verarbeitung und ggf. Logging oder Anzeige (OLED)
  \end{itemize}
\end{itemize}

\section*{Vorteile}
\begin{itemize}
  \item Keine empfindlichen Kabelverbindungen zwischen Deckel und Boden
  \item Keine Stolpergefahr beim Imkern
  \item Modularität: Komponenten lassen sich einfacher austauschen
  \item Sehr stromsparend: Nur zyklischer Funkversand der Sensordaten
\end{itemize}

\section*{Offene Fragen}
\begin{itemize}
  \item Welche Stromversorgung ist für das Deckelmodul langfristig am stabilsten?
  \item Wie oft sollen Datenpakete gesendet werden (Intervall)?
  \item Ist ggf. eine kleine Zwischenspeicherung der Sensordaten sinnvoll?
\end{itemize}

\section*{Fazit}
Die Trennung von Sensorik (oben) und Aktorik (unten) über ein minimalistisches Funkprotokoll wie ESP-NOW erscheint praktikabel und elegant. Der Einsatz eines zweiten ESP32 ist durch die geringen Kosten mehr als vertretbar und ermöglicht eine klare strukturelle Trennung der Aufgaben.

\end{document}
