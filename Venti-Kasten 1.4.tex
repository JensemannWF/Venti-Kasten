\documentclass[11pt,a4paper]{article}
\usepackage[utf8]{inputenc}
\usepackage[T1]{fontenc}
\usepackage{lmodern}
\usepackage[left=2.5cm,right=2.5cm,top=2.5cm,bottom=2.5cm]{geometry}
\usepackage{enumitem}
\usepackage{hyperref}
\hypersetup{colorlinks=true, linkcolor=blue, urlcolor=blue}

\title{\textbf{Venti-Kasten 1.4 – Luftzirkulation über U-Flow-Prinzip (Gaze-Boden)}}
\author{Jens Buttenschön}
\date{30. April 2025}

\begin{document}

\maketitle

\section*{Überblick}

Version 1.4 erweitert das bestehende Konzept des Venti-Kastens um eine sanfte, strömungstechnisch unterstützte Luftführung zur Entfeuchtung des Bienenstocks. Statt aktiver Eingriffe in den Beutenaufbau wird der vorhandene Gaze-Boden gezielt genutzt, um eine natürliche Konvektionsbewegung zu fördern. Die Luft wird unten links eingebracht und über einen modifizierten Deckel kontrolliert wieder abgeführt.

\section*{U-Flow-Zirkulation über bestehenden Gaze-Boden}

\begin{itemize}[leftmargin=1.5em]
  \item \textbf{Einblasung:} Ein Lüftermodul wird von außen auf das linke Drittel des Gaze-Bodens gesetzt. Die restliche Struktur der Beute bleibt unangetastet.
  \item \textbf{Luftführung:} Die eingeblasene Luft steigt links durch die Wabengassen nach oben und wird im Deckel horizontal nach rechts geleitet.
  \item \textbf{Deckelstruktur:} Der Standarddeckel wird innen um ca. 15–20 mm vertieft und erhält Gaze-Einsätze auf der linken und rechten Seite. Die Mitte bleibt geschlossen, wodurch der Luftstrom gezielt gelenkt wird.
  \item \textbf{Ausleitung:} Die feuchte Luft verlässt die Beute über das rechte Drittel der Boden-Gaze nach unten – ohne aktiven Abzug, sondern durch Druckausgleich.
\end{itemize}

\section*{Komponenten}

\begin{itemize}[leftmargin=1.5em]
  \item \textbf{Lüftereinheit:} Extern anbringbares Modul mit Walzen- oder Axiallüfter auf dem linken Drittel der Boden-Gaze.
  \item \textbf{Deckeleinheit:} Sensorik (Temperatur, Feuchtigkeit) + Steuerung + PV-Stromversorgung.
  \item \textbf{Verkabelung:} Durchführung durch das Styropor von außen (z. B. mit Nadel vorgestochen). Keine interne Verdrahtung im Brut- oder Honigraum.
\end{itemize}

\section*{Vorteile}

\begin{itemize}[leftmargin=1.5em]
  \item Keine Eingriffe in Honig- oder Brutraum nötig
  \item Nutzung vorhandener Gaze-Bodenstruktur
  \item Kompatibel mit Standard-Sägeberger Deckel
  \item Geringes Gewicht, wartungsarm, wettergeschützt
  \item Nachrüstbar für bestehende Systeme
\end{itemize}

\section*{Ausblick}

Diese U-Flow-Lösung stellt eine strömungslogisch unterstützte, aber biologisch angepasste Variante dar, wie moderne Belüftungstechnologie mit der natürlichen Architektur der Beute kombiniert werden kann – ohne in das Innenleben der Bienenkolonie einzugreifen.

\end{document}