\documentclass[a4paper,11pt]{article}
\usepackage[utf8]{inputenc}
\usepackage{lmodern}
\usepackage{geometry}
\usepackage{amsmath}
\usepackage{graphicx}
\usepackage{enumitem}
\usepackage{hyperref}

\geometry{margin=2.5cm}

\title{Technische Maße: Einschubmodul für Venti-Kasten}
\author{Jens Buttenschön}
\date{Mai 2025}

\begin{document}

\maketitle

\section*{Übersicht}
Die nachfolgenden Maße beziehen sich auf den Einschubbereich unterhalb des Bodens einer Segeberger Beute, der ursprünglich für die Varroa-Windel vorgesehen ist. Dieser Bereich wird für den Einbau eines aktiven Belüftungsmoduls (Venti-Kasten-Einschub) umfunktioniert.

\section*{Maße der Einschuböffnung}
\begin{itemize}[label=--]
  \item \textbf{Maximale Einschubhöhe (lichte Weite der Schiene):} 24\,mm
  \item \textbf{Aktuelle Plattendicke (Originalwindel):} 23\,mm
  \item \textbf{Maximale nutzbare Tiefe (Einstecktiefe):} 432\,mm
  \item \textbf{Breite der Platte (zwischen den Führungsschienen):} 311\,mm
\end{itemize}

\section*{Freiraum unter der eingeschobenen Windel}
\begin{itemize}[label=--]
  \item \textbf{Lichte Höhe bei offener Öffnung:} 60\,mm
  \item \textbf{Freiraum unter Windel nach Einschub:} 36\,mm
  \item \textbf{Potenzial durch dünnere Trägerplatte:} bis ca. 50\,mm nutzbare Bauhöhe
\end{itemize}

\section*{Geplante Komponenten im Einschub}
\begin{itemize}[label=--]
  \item \textbf{2 × 80\,mm Lüfter} (axial, oben blasend, rechts positioniert)
  \item \textbf{Vertikale Trennplatte} zur Luftkanalisierung, Höhe bis 58\,mm (Bodenfreiheit 2\,mm)
  \item \textbf{Möglichkeit zur horizontalen Luftführung} durch hintere Auslassöffnung
\end{itemize}

\section*{Hinweis}
Die Maße wurden an einer typischen Segeberger Kunststoffbeute mit Varroa-Windel gemessen und können leicht variieren. Toleranzen sind beim Bau zu berücksichtigen.

\end{document}