\documentclass[11pt,a4paper]{article}
\usepackage[utf8]{inputenc}
\usepackage[T1]{fontenc}
\usepackage{lmodern}
\usepackage{geometry}
\usepackage{titlesec}
\usepackage{enumitem}
\usepackage{hyperref}

\geometry{margin=2.5cm}
\titleformat{\section}{\normalfont\Large\bfseries}{\thesection}{1em}{}
\titleformat{\subsection}{\normalfont\bfseries}{\thesubsection}{1em}{}

\title{\parbox{\linewidth}{\centering\textbf{
Projektidee: Der virtuelle Naturbaum\\
Adaptive Konvektionsl\"uftung f\"ur Bienenbeuten (Venti-Kasten 1.0)}}}

\author{\parbox{\linewidth}{\centering
Jens Buttensch\"on}}

\date{April 2025 -- Version 1.0}

\begin{document}
\maketitle

\section*{Leitgedanke: Der virtuelle Naturbaum}
In der freien Natur w\"ahlen Bienen bevorzugt hohle B\"aume als Behausung. Diese bieten ein
stabiles, selbstregulierendes Mikroklima: Holz puffert Feuchtigkeit, Konvektion sorgt f\"ur sanften
Luftaustausch, ohne dass die Bienen aktiv eingreifen m\"ussen. Dieses Prinzip simulieren wir
technologisch -- und erschaffen einen virtuellen Naturbaum innerhalb der modernen Beute:
energieeffizient, ressourcenschonend, und im besten Sinne \"okologisch smart.

\section{Zielsetzung}
Entwicklung eines autarken L\"uftungssystems, das auf nat\"urlicher Konvektion basiert und gezielt
die Luftfeuchtigkeit und Temperatur im Honigraum reguliert, um:
\begin{itemize}
  \item den F\"achelaufwand der Bienen zu minimieren,
  \item die Honigreife zu beschleunigen,
  \item Temperaturschwankungen im Innenraum auszugleichen,
  \item ein baumh\"ohlen\"ahnliches Mikroklima zu schaffen.
\end{itemize}

\section{Technisches Prinzip: Der umfunktionierte Futterkasten}
Die klassische Futterzarge wird zur Klimasteuerzentrale:
\begin{itemize}
  \item \textbf{Sensorik:} Temperatur- und Luftfeuchtesensor im oberen Beutenbereich.
  \item \textbf{Aktoreinheit:} Servo- oder Schwenkklappe zur Abluft\"offnung im Deckelbereich.
  \item \textbf{Logik:} Mikrocontroller (z.\,B. ESP32/Arduino) mit adaptiver Regelsoftware.
  \item \textbf{Energie:} Versorgung \"uber PV und Akkupuffer.
\end{itemize}

\section{Adaptiver Algorithmus}
Die Software erfasst zyklisch Innenklimawerte und steuert die Klappen\"offnung in Abh\"angigkeit von:
\begin{itemize}
  \item relativer Luftfeuchtigkeit ($\Delta$rF/$\Delta$t),
  \item Innen- und Au\ss{}entemperatur,
  \item Feuchte- und Temperaturprofil der Tageszeiten.
\end{itemize}
Durch maschinelles Lernen lassen sich sp\"ater optimierte L\"uftungs- und Temperaturphasen modellieren.

\section{Nachhaltigkeit und Wirkung}
Der Venti-Kasten 1.0 ist ein Beispiel f\"ur smarte \"Okotechnologie in der modernen Imkerei:
\begin{itemize}
  \item \textbf{ressourcenschonend:} nutzt vorhandene W\"arme- und Str\"omungsprinzipien,
  \item \textbf{energieeffizient:} kein mechanisches L\"uftungssystem n\"otig,
  \item \textbf{bienenfreundlich:} weniger Energieverbrauch f\"ur F\"acheln und Stabilisierung,
  \item \textbf{\"okologisch intelligent:} sowohl f\"ur Styropor- als auch Holzbeuten geeignet.
\end{itemize}

\section{N\"achste Schritte}
\begin{enumerate}
  \item Entwurf eines modularen Ventikasten-Prototyps.
  \item Integration der L\"uftungs- und Temperaturregelung in vorhandene Sensoriksysteme.
  \item Pilotversuch an zwei Beuten unter realen Au\ss{}enbedingungen.
\end{enumerate}

\section*{Kontakt}
\textbf{Jens Buttensch\"on} \\
\small{\url{https://www.xing.com/profile/Jens_Buttenschoen}}

\end{document}
