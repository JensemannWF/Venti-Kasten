\documentclass[11pt,a4paper]{article}
\usepackage[utf8]{inputenc}
\usepackage[T1]{fontenc}
\usepackage{lmodern}
\usepackage{geometry}
\usepackage{titlesec}
\usepackage{enumitem}
\usepackage{hyperref}

\geometry{margin=2.5cm}
\titleformat{\section}{\normalfont\Large\bfseries}{\thesection}{1em}{}
\titleformat{\subsection}{\normalfont\bfseries}{\thesubsection}{1em}{}

\title{\parbox{\linewidth}{\centering\textbf{
Projektidee: Der virtuelle Naturbaum\\
Adaptive Konvektionslüftung für Bienenbeuten (Venti-Kasten 1.0)}}}

\author{\parbox{\linewidth}{\centering
Jens Buttenschön\\
\small in Kooperation mit Marc Juncker, Mittelstand-Digital Zentrum Hannover\\
(Bundesministerium für Wirtschaft und Klimaschutz), Standort Garbsen}}

\date{April 2025 – Version 1.0}

\begin{document}
\maketitle

\section*{Leitgedanke: Der virtuelle Naturbaum}
In der freien Natur wählen Bienen bevorzugt hohle Bäume als Behausung. Diese bieten ein stabiles, selbstregulierendes Mikroklima: Holz puffert Feuchtigkeit, Konvektion sorgt für sanften Luftaustausch, ohne dass die Bienen aktiv eingreifen müssen. 
Dieses Prinzip simulieren wir technologisch – und erschaffen einen \textbf{virtuellen Naturbaum} innerhalb der modernen Beute: energieeffizient, ressourcenschonend, und im besten Sinne ökologisch smart.

\section{Zielsetzung}
Entwicklung eines autarken Lüftungssystems, das auf natürlicher Konvektion basiert und gezielt die Luftfeuchtigkeit und Temperatur im Honigraum reguliert, um:
\begin{itemize}[topsep=2pt]
  \item den Fächelaufwand der Bienen zu minimieren,
  \item die Honigreife zu beschleunigen,
  \item Temperaturschwankungen im Innenraum auszugleichen,
  \item ein baumhöhlenähnliches Mikroklima zu schaffen.
\end{itemize}

\section{Technisches Prinzip: Der umfunktionierte Futterkasten}
Die klassische Futterzarge wird zur \textbf{Klimasteuerzentrale}:
\begin{itemize}[topsep=2pt]
  \item \textbf{Sensorik:} Temperatur- und Luftfeuchtesensor im oberen Beutenbereich.
  \item \textbf{Aktoreinheit:} Servo- oder Schwenkklappe zur Abluftöffnung im Deckelbereich.
  \item \textbf{Logik:} Mikrocontroller (z.\,B. ESP32/Arduino) mit adaptiver Regelsoftware.
  \item \textbf{Energie:} Versorgung über PV und Akkupuffer.
\end{itemize}

\section{Adaptiver Algorithmus}
Die Software erfasst zyklisch Innenklimawerte und steuert die Klappenöffnung in Abhängigkeit von:
\begin{itemize}[topsep=2pt]
  \item relativer Luftfeuchtigkeit ($\Delta$rF/$\Delta$t),
  \item Innen- und Außentemperatur,
  \item Feuchte- und Temperaturprofil der Tageszeiten.
\end{itemize}
Durch maschinelles Lernen lassen sich später optimierte Lüftungs- und Temperaturphasen modellieren.

\section{Integration mit KI-Infrastruktur}
Dank der bestehenden Systeme von \textbf{Marc Juncker (Mittelstand-Digital Zentrum Hannover)} ergeben sich exzellente Synergien:
\begin{itemize}[topsep=2pt]
  \item \textbf{Stromversorgung:} Nutzung vorhandener PV-/Akkuinfrastruktur.
  \item \textbf{Kommunikation:} Datenanbindung via Mobilfunkchip (bestehende Lösung).
  \item \textbf{Sensorfusion:} Gewicht, Wetter, Aktivität und Klimawerte kombinierbar.
  \item \textbf{KI-Prognose:} Honigreifemuster, Lüftungsbedarf und Wärmelasten vorhersagbar.
\end{itemize}

\section{Nachhaltigkeit und Wirkung}
Der Venti-Kasten 1.0 ist ein Beispiel für smarte Ökotechnologie in der modernen Imkerei:
\begin{itemize}[topsep=2pt]
  \item \textbf{ressourcenschonend:} nutzt vorhandene Wärme- und Strömungsprinzipien
  \item \textbf{energieeffizient:} kein mechanisches Lüftungssystem nötig
  \item \textbf{bienenfreundlich:} weniger Energieverbrauch für Fächeln und Stabilisierung
  \item \textbf{ökologisch intelligent:} sowohl für Styropor- als auch Holzbeuten geeignet
\end{itemize}

\section{Nächste Schritte}
\begin{enumerate}[topsep=2pt]
  \item Rücksprache mit Marc zu Hardwareschnittstellen und Energiekapazitäten.
  \item Entwurf eines modularen Ventikasten-Prototyps.
  \item Integration der Lüftungs- und Temperaturregelung in bestehende KI-Schnittstellen.
  \item Pilotversuch an zwei Beuten unter realen Außenbedingungen.
\end{enumerate}

\section*{Kontakt}
\textbf{Jens Buttenschön} \\
\small{\url{https://www.xing.com/profile/Jens_Buttenschoen}}

\end{document}